% LaTeX-Vorlage für die Titelseite und Selbständigkeitserklärung einer Abschlussarbeit
% basierend auf der vorigen Institutsvorlage des Instituts für Informatik
% sowie der Vorlage für Promotionsarbeiten.
%
% erweitert: 2014-06-12 Dennis Schneider <dschneid@informatik.hu-berlin.de>

% gepunktete Linie unter Objekt:
\newcommand{\TitelPunkte}[1]{%
  \tikz[baseline=(todotted.base)]{
    \node[inner sep=1pt,outer sep=0pt] (todotted) {#1};
    \draw[dotted] (todotted.south west) -- (todotted.south east);
  }%
}%

% gepunktete Linie mit gegebener Länge:
\newcommand{\TitelPunktLinie}[1]{\TitelPunkte{\makebox[#1][l]{}}}

\makeatletter

\newcommand*{\@titelTitel}{Titel der Arbeit}
\newcommand{\titel}[1]{\renewcommand*{\@titelTitel}{#1}} % Titel der Arbeit
\newcommand*{\@titelArbeit}{Arbeitstyp}
\newcommand{\typ}[1]{\renewcommand*{\@titelArbeit}{#1}} % Typ der Arbeit
\newcommand*{\@titelGrad}{akademischer Grad}
\newcommand{\grad}[1]{\renewcommand*{\@titelGrad}{#1}} % Akademischer Grad
\newcommand*{\@titelAutor}{Autor}
\newcommand{\autor}[1]{\renewcommand*{\@titelAutor}{#1}} % Autor der Arbeit
\newcommand*{\@titelGeburtsdatum}{\TitelPunktLinie{2cm}}
\newcommand{\gebdatum}[1]{\renewcommand*{\@titelGeburtsdatum}{#1}} % Geburtsdatum des Autors
\newcommand*{\@titelGeburtsort}{\TitelPunktLinie{5cm}}
\newcommand{\gebort}[1]{\renewcommand*{\@titelGeburtsort}{#1}} % Geburtsort des Autors
\newcommand*{\@titelGutachterA}{\TitelPunktLinie{5cm}}
\newcommand*{\@titelGutachterB}{\TitelPunktLinie{5cm}}
\newcommand{\gutachter}[2]{\renewcommand*{\@titelGutachterA}{#1}\renewcommand*{\@titelGutachterB}{#2}} % Erst- und Zweitgutachter
\newcommand*{\@titelEinreichungsdatum}{\TitelPunktLinie{3cm}} % Datum der Einreichung, wird nicht vom Studenten ausgefüllt
\newcommand*{\@titelVerteidigungsdatum}{} % Verteidigungstext, wird nicht vom Studenten ausgefüllt
\newcommand{\mitverteidigung}{\renewcommand*{\@titelVerteidigungsdatum}{verteidigt am: \,\,\TitelPunktLinie{3cm}}} % Verteidigungsplatzhalter erzeugen
\newcommand*{\@wastwoside}{}

% Titelseite erzeugen:
\newcommand{\makeTitel}{%
	% Speichere, ob doppelseitiges Layout gewählt wurde:
\if@twoside%
	\renewcommand*{\@wastwoside}{twoside}
\else
	\renewcommand*{\@wastwoside}{twoside=false}
\fi
	\KOMAoptions{twoside = false}% Erzwinge einseitiges Layout (erzeugt eine Warnung)

	\begin{titlepage}
		% Ändern der Einrückungen
		\newlength{\parindentbak} \setlength{\parindentbak}{\parindent}
		\newlength{\parskipbak} \setlength{\parskipbak}{\parskip}
		\setlength{\parindent}{0pt}
		\setlength{\parskip}{\baselineskip}

		\thispagestyle{empty}

		\begin{minipage}[c][3cm][c]{12cm}
			\textsc{%
				% optischer Randausgleich per Hand:
				\hspace{-0.4mm}\textls*[68]{\Large Humboldt-Universität zu Berlin}\\
				\normalsize \textls*[45]{
					Mathematisch-Naturwissenschaftliche Fakultät\\
					Institut für Informatik
				}
			}
		\end{minipage}
		\hfill


		% Also wenn schon serifenlose Schriften (Titel), dann ganz oder gar nicht
		\sffamily

		\vfill

		\begin{center}
		\begin{doublespace}
			\vspace{\baselineskip}
			{\LARGE \textbf{\@titelTitel}}\\
			%\vspace{1\baselineskip}
			{\Large
				\@titelArbeit\\
				zur Erlangung des akademischen Grades\\
				\@titelGrad
				\vspace{\baselineskip}
			}
		\end{doublespace}
		\end{center}

		\vfill
\newcolumntype{L}{>{\raggedright\arraybackslash}X}
		{\large \raggedleft
			\begin{tabularx}{\textwidth}{l@{\,\,\raggedright~}L} % verbreiterter Abstand zwischen Feldern wurde gewünscht
				eingereicht von: & \@titelAutor\\
				geboren am: & {\@titelGeburtsdatum}\\
				geboren in: & \@titelGeburtsort
				\vspace{0.5\baselineskip}\\
				Gutachter/innen: & \@titelGutachterA \\
					& \@titelGutachterB
				\vspace{0.5\baselineskip}\\
				eingereicht am: & \@titelEinreichungsdatum \hfill \@titelVerteidigungsdatum
			\end{tabularx}}
			\vspace{-1\baselineskip}\\\phantom{x} % Übler Hack, um eine Warnung wg. einer zu leeren hbox zu verhindern
		% Wiederherstellen der Einrückung
		\setlength{\parindent}{\parindentbak}
		\setlength{\parskip}{\parskipbak}
	\end{titlepage}

	% Aufräumen:
	\let\@titelTitel\undefined
	\let\titel\undefined
	\let\@titelArbeit\undefined
	\let\typ\undefined
	\let\@titelGrad\undefined
	\let\grad\undefined
	\let\@titelAutor\undefined
	\let\autor\undefined
	\let\@titelGeburtsdatum\undefined
	\let\gebdatum\undefined
	\let\@titelGeburtsort\undefined
	\let\gebort\undefined
	\let\@titelGutachterA\undefined
	\let\@titelGutachterB\undefined
	\let\gutachter\undefined
	\let\@titelEinreichungsdatum\undefined
	\let\einreichungsdatum\undefined
	\let\@titelVerteidigungsdatum\undefined
	\let\verteidigungsdatum\undefined
	\let\@titelBetreuer\undefined
	\let\betreuer\undefined

	\KOMAoptions{\@wastwoside}% Stelle alten Modus (ein-/doppelseitig) wieder her
	\let\@wastwoside\undefined
	\cleardoublepage % ganzes Blatt für die Titelseite
}

% Als Allerallerletztes kommt Selbständigkeitserklärung:
% Aufruf mit dem Datum in deutscher und englischer Form
\newcommand{\selbstaendigkeitserklaerung}[1]{%
	\cleardoublepage% Wieder auf eine eigene Doppelseite
	\thispagestyle{plain}

	{\parindent0cm
		\subsection*{Selbständigkeitserklärung}
		Ich erkläre hiermit, dass ich die vorliegende Arbeit selbständig verfasst
		und noch nicht für andere Prüfungen eingereicht habe.
		Sämtliche Quellen einschließlich Internetquellen, die unverändert oder
		abgewandelt wiedergegeben werden, insbesondere Quellen für Texte, Grafiken,
		Tabellen und Bilder, sind als solche kenntlich gemacht. Mir ist bekannt,
		dass bei Verstößen gegen diese Grundsätze ein Verfahren wegen
		Täuschungsversuchs bzw. Täuschung eingeleitet wird.
		\vspace{3\baselineskip}

		{\raggedright Berlin, den #1 \hfill \TitelPunktLinie{8cm}\\}
%		\vspace{3\baselineskip}
%
% 		\selectlanguage{english}
% 		\subsection*{Statement of authorship}
% 		Hier würde die englische Selbständigkeitserklärung folgen, falls gewünscht. Doch es fehlt eine akzeptable Übersetzung.
% 		\vspace{3\baselineskip}
%
% 		Berlin, #2 \hfill \TitelPunktLinie{6cm}
	}
}%

\makeatother
