\section{Introduction}
\label{sec:introduction}

\subsection{Motivation}
Many universities require students to enroll in a seminar in order to obtain their degree. Usually the students have a choice between a handful of different seminars, however there are capacity constraints that make it hard to give all students their first choice. Let's consider the following example: 100 students have to be assigned to one of 6 seminars, where each of the seminars has a capacity of 20. The students express their preferences by supplying a strict, but incomplete preference list of the 6 seminars. The goal for the school's administration now is to assign as many students as possible to a seminar of their choice. 
\newline
What makes this problem harder is that the students preferences aren't necessarily equally distributed. Oftentimes a majority of students prefers one seminar in which case conflicts exist in choosing which students get their first choice. At the same time, it can happen that students go unmatched if their preference lists are short and full of seminars that have reached full capacity already.
\newline
The goal of this thesis is to formally model the aforementioned problem, while presenting different algorithms for finding possible matchings and finally to evaluate these algorithms using certain metrics that make sense in the domain of matching, such as stability, rank-maximality, popularity and pareto-optimality. Additionally, an interactive system will be developed, which allows a school's administration to find a student-seminar matching using one of the presented algorithms. 

\subsection{Formal Definition}
The problem of assigning students to seminars can be described as a many-to-one matching, with a set of students $S:= \{s_1, s_2, ...,s_n\}$ and a set of seminars $T:= \{t_1, t_2, ..., t_m\}$. Every student $s_i$ provides a strict preference order over a subset of $T$, and every seminar $t_j$ has a capacity of $c_j$ students. The goal is to find a matching $M: S \rightarrow T$, that assings students to their preferred seminars while respecting the capacity of the seminars. That means that for every seminar $t_j$ the following is satisfied: $|M(t_j)| \leq c_j$. Using this definition we can describe the problem as a many-to-one matching with one-sided, incomplete preferences.  

\subsection{Similar Problems}
There are many similar problems, which differ in what the preference lists look like and how many entities of the first set are matched to how many entities of the second set. Studying of these problems reveals some important insights into matching markets.

\subsubsection{Stable Marriage Problem}
The stable marriage problem was one of the first matching problems to be researched\cite{GaleShapleyOrig} and consequently motivated a lot more research in the domain of matching.
\newline
The problem is stated as follows: A set of $m$ men and $n$ women shall be matched one-to-one, where each men and women provide a complete strict-preference order over the agents of the other set. The deferred acceptance algorithm presented in Gale and Shapley's paper\cite{GaleShapleyOrig}
finds a stable, complete matching in polynomial time. Stability is defined as follows: given a women $w$ and any man that she was not matched to $m$, $w$ does not prefer $m$ more than her current partner, and $m$ does not prefer $w$ more than his current partner. 
\newline
TODO: present algorithm here?
\newline
The algorithm can be executed in two ways: 
\begin{enumerate}
    \item the men have priority, by proposing to women, where the woman has to accept the proposal iff it improves her situation.  
    \item the women have priority and propose to men. This case is analogue to the first one
\end{enumerate}
It has been shown that all possible executions of the algorithm with men as proposers yield the same stable matching. That matching is men-optimal, which means that every man has the best partner that he can have in any stable matching.\cite{Gusfield} Additionally, with men proposing the produced matchings has also been shown to be women-pessimal, meaning that every woman is matched to the worst partner that she can have in any matching.\cite{Gusfield}
\newline
While the scenario of matching women to men like described (hopefully) doesn't occur in the real world, Gale and Shapley's paper inspired a lot more approaches to other practical problems:

\subsubsection{The Hospitals/Residents Problem}
A few years after the publication of Gale and Shapley's original paper, it was found that the deferred acceptance algorithm was essentially the same algorithm used by the National Resident Matching Program (NRMP) in the United States to match graduating medical students to residency positions in hospitals.\cite{Gusfield} As a matter of fact, Gale and Shapley's paper described an algorithm for the so-called "college admissions problem"\cite{GaleShapleyOrig}, which is essentially the same problem. Just like in the stable-marriage problem, there is a solution that be hospital-optimal or resident-pessimal. 
\newline
The problem can be described as finding a one-to-many matching with two-sided incomplete, but strict preferences. Essentially, a hospital can offer multiple spots and both parties can mark entities from the other set as unacceptable by not including them in their preference list.\cite{RePEc:ris:nobelp:2012_005} 
\newline
In reality, the problem is a bit more complex, as it permits couples of residents to submit preferences together. It has been shown that a stable solution does not always exist and that finding one if it exists or showing that it doesn't exist is NP-complete.\cite{RONN1990285} The revised algorithm used by the NRMP uses findings about stability and simple matching markets to find a good approximation, while minimizing opportunities for strategic manipulation, which was indeed possible before.\cite{NBERw6963}

\subsubsection{House allocation problem}
Many economists and game theorists\cite{FEKETE2003219} have studied variants of the house allocation (HA) problem, where a set of indivisible items $H$ needs to be divided among a set $A$ of applicants. Each one of those applicants may have a strict preference order over a subset of $H$. Formally this means that an instance $I$ of the problem consists of two disjoint sets, where $H := \{h_1, h_2, ..., h_n\}$ is the set of houses and $A := \{a_1, a_2, ..., a_m\}$ is the set of applicants. Each applicant $a_i \in A$ ranks a subset of the houses in $H$ using a preference list. The houses on the other hand do not have any preferences over applicants. A matching or assignment $M$ is a subset of $A \times h$, so that for every applicant $a_i$, the house $M(a_i)$ is indeed on the applicants preference list.\cite{sng-matching} 
\newline
The house allocation problem is essentially an alias for a matching problem on bipartite graphs with one-sided preferences. There are many applications including matching clients to servers, professors to offices and also students to seminars. For the latter some generalisations have to be made to the problem. Specifically one seminar should now be matched to more than one student, whereas the houses in HA are matched to one and only one applicant. In the literature that variant of the problem is also referred to as the Capacitated House Allocation Problem, denoted by CHA.\cite{algorithmics} (TODO) The next section will explore performance indicators and algorithms for finding such matchings in the context of student-seminar matchings.

\subsubsection{Assignment Problem}
The problem of matching students to seminars can also be defined as an assignment problem. The goal of the assignment problem is to find a minimum weight perfect matching in a bipartite graph. In this case the goal of the problem is, given the set of students $A$, seminars $B$, and a cost function $W: A \times B \rightarrow I\!R$, to find a map $M: A \rightarrow B$, which minimzes the following objective function:
$\sum_{a \in A} W(a, M(a))$.
\newline
One of the first algorithms used for solving this problem was the Hungarian algorithm by Munkres, which finds a minimum-weight matching in polynomial time.\cite{Munkres} 
\newline
Alternatively the problem can be transformed into an instance of the minimum-cost flow problem to determine a minimum weight matching. Section 3.4 (TODO) will further investigate using this algorithm for the problem of student-seminar matchings.

\subsection{Optimality criteria for matchings}
Looking at the previously presented problems, it is clear that there are different objective functions or optimality criteria for such matchings. For instance, in the stable marriage problem, matchings are primarily judged by the stability characteristic. However such a characteristic does not make sense in the case of the student-seminar problem, since it assumes both-sided preferences. For that very reason, different criteria have to be used for judging the quality of a matching.
\newline
Intuitively, when thinking about matching students to seminars, it would be desirable to match as many students as possible, as well as matching the students to their first choice, etc. To formalize these requirements, a few criteria have been discussed in academia, which will be helpful for comparing different approaches. 

\subsubsection{Maximum cardinality}
The goal of the maximum cardinality problem is finding a matching $M$ on a graph $G=(V=(X, Y), E)$, so that $|M|$ is maximal.\cite{GraphTheoryIntro} Consequently, maximum cardinality as an optimality criteria means that a matching is ideal in the sense of the criteria if the number of students that are matched is maximized among all possible matchings.

\subsubsection{Popularity}

\subsubsection{Pareto Optimality}
\subsubsection{Rank Maximality}
\subsubsection{Profile-based Optimality}
\subsubsection{Maximum Utility}

\subsection{Outline}
Now that the problem has been formalized and similar problems have been presented, I will present several different algorithmic approaches to find possible matchings. To evaluate these matchings I will also present commonly used metrics like:
\begin{itemize}[itemsep=0pt]
    \item Rank-maximality (maximum number of students ranked to their first priority)
    \item Maximum cardinality
    \item Pareto optimality
    \item Popularity 
    \item Profile-based optimality
\end{itemize}
Using these metrics I will choose one algorithm for implementation and analyze it's space and time complexities as well as evaluating it's performance using the aforementioned metrics against the other approaches. Next, I will describe the interactive web system developed for using the algorithm and lastly extend the problem to a two-sided problem. 