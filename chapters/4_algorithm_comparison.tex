\subsection{Comparison of Mechanisms}
In the previous section, we studied several different algorithmic approaches that guarantee different optimality criteria. To better evaluate the algorithms against the optimality criteria we defined in Section \ref{sec:optimality} we will now summarize and compare the properties of the algorithms. Afterwards, we will look at practical results that were obtained by performing experiments of matching mechanisms with one- and two-sided preferences by Diebold and Bichler \cite{DieboldBenchmark}.

\subsubsection{Theoretical Results}
Referring back to the list of desirable properties defined in Section \ref{criteria-application}, let us now recap and compare the aforementioned algorithms to evaluate which one could be applicable for the problem of matching students to seminars. Unfortunately, none of the algorithms guarantee all of the optimality criteria at the same time, which makes the choice of an algorithm non-trivial. Table \ref{tab:algorithm-comparison} gives an overview of the presented algorithms and their properties. Each of the algorithms mentioned in this section is listed, and for each optimality criteria, a yes/no encoding is used to make a statement about which properties an algorithm guarantees. It is important to note here that a "no" in a column does not strictly mean that the given optimality criteria cannot be fulfilled by the algorithm, but rather that the algorithm does not guarantee it. For instance, a matching computed with the greedy algorithm can be of maximum cardinality or be popular. Only the results for strategy-proofness are a strict yes or no, since fulfilling strategy-proofness does not depend on the instance of the problem, but only of the mechanism being used. 

\begin{table}[h!]
    \begin{tabular}{l|llll}
    \hline
                        & RSD    & Max-PaCHA    & Assignment & Popular-CHA           \\ \hline
    Maximum Cardinality & no     & yes          & yes        & yes               \\
    Pareto-Optimal      & yes    & yes          & yes        & yes               \\
    Popular             & no     & no           & no         & yes               \\
    Rank Maximal        & no     & no           & yes        & no                \\
    Always Exists       & yes    & yes          & yes        & no                \\
    Strategy Proof      & yes    & no           & no         & yes               \\ \hline
    Time Complexity     & $\mathcal{O}(n)$   & $\mathcal{O}(\sqrt{n} * m)$ & $\approx\mathcal{O}(n^3)$    & $\mathcal{O}(\sqrt{C} * n_1 + m)$ \\ \hline
    \end{tabular}
    \caption{Comparison of different algorithmic approaches}
    \label{tab:algorithm-comparison}
\end{table}

To summarize the results, we can see that all of the algorithms guarantee pareto-optimality, however only Popular-CHA guarantees popularity. At the same time, only RSD and Popular-CHA also guarantee strategy-proofness, which makes Popular-CHA particularly interesting for the student-seminar problem. 

\paragraph{Strategy-proofness and Maximum Cardinality:}
One interesting observation is that fulfilling maximum cardinality comes at the cost of either not being strategy-proof, or not guaranteeing that a matching exists at all. Indeed, only RSD and Popular-CHA algorithm guarantee strategy-proofness. However, ensuring strategy-proofness and maximum cardinality at the same time comes at the cost of not always finding a matching (Popular CHA). If we look back at the Popular-CHA algorithm, we remember that a maximum cardinality matching $M'$ is computed on the reduced graph $G'$. We saw that a maximum popular matching does not exist, iff the matching is not agent-complete, meaning that one of the agents is matched to their last-resort house. While this mechanism ensures strategy-proofness, it is also not always possible to find such a maximum cardinality matching using the Popular-CHA algorithm. Therefore, it remains an open question whether or not a mechanism exists that both is strategy-proof and always produces maximum-cardinality matchings.

\paragraph{Max-PaCHA and the Assignment Problem:}
Another important thing to notice is the similarity of the properties between the Max-PaCHA and assignment problem algorithm. Except for the fact that the assignment algorithm guarantees rank maximality, the two algorithms produce matchings with very similar characteristics, which then begs the question of why one should use the Max-PaCHA algorithm. But looking at the runtime complexity of the algorithms, we see that, while both algorithms run in polynomial time, the assignment problem takes longer to be solved.

\subsubsection{Practical Results in the Literature}\label{sec:practical-results-lit}
Diebold et al. have published results for extensive experiments on matching mechanisms with both one- and two-sided preferences \cite{DieboldBenchmark}. They used real course registration data from TUM for investigating properties, including size, rank and popularity, of matchings produced by several mechanisms. The mechanisms are the same as described in Section \ref{chapter:algorithms}, with one exception being that the \emph{ProB-CHAT} algorithm \cite{DieboldBenchmark} is used in place of the Hungarian algorithm for finding rank-maximal matchings.

The authors found that all algorithms' matchings achieve an average cardinality of at least 97.48\%. For one of the 9 instances, Popular-CHA failed to find a matching, but when it found one, the matchings' average ranks of 1.33 got close to the ones produced by ProB-CHAT of 1.26. Unsurprisingly, ProB-CHAT performed best on the rank metrics and also produced more popular matchings than all other algorithms except for Popular-CHA. However, its max runtime was the worst with 33.852s, compared to the second highest 2.458s of Popular-CHA on a dataset with 915 students and 51 courses.

Another surprising finding is that RSD performed better on rank metrics with an average rank of 1.41 than Max-PaCHA with an average rank of 1.51. Generally, the average ranks of all algorithms are somewhat close by being in a range of 1.26 (ProB-CHAT) to 1.51 (Max-PaCHA). Unfortunately, the authors do not disclose more detailed information about their datasets and structure of matchings. 

While these results confirm some of the theoretical observations, it will be interesting to get more insights with differently structured datasets being used. Additionally, most of the data contains ties and we do not get any meaningful insights on the distribution of preferences. 
