\section{Comparison of mechanisms}

\subsection{Theoretical results}
Drawing back to the list of desirable properties defined in section \ref{criteria-application}, let us now recap and compare the aforementioned algorithms to evaluate which one could be applicable for the problem of matching students to seminars. Unfortunately none of the algorithms guarantee all of the optimality critera at the same time, which makes the choice of an algorithm unclear. Table \ref{tab:algorithm-comparison} gives an overview of the presented algorithms and their properties. Each of the algorithms is listed in the same order that they were presented, and for each optimality criteria, a yes/no encoding is used to make a statement about which properties an algorithm guarantees. It is important to note here that a "no" in a column does not strictly mean that the given optimality criteria cannot be fulfilled by the algorithm, but rather that the algorithm does not guarantee it. For instance, a matching computed with the greedy algorithm can be of maximum cardinality or be popular. Only the results for strategy-proofness are a strict yes or no, since fulfilling strategy-proofness does not depend on the instance of the problem, but only of the mechanism being used. 

\begin{table}[h!]
    \begin{tabular}{lllll}
    \hline
                        & Greedy & Max Pareto   & Assignment & Popular           \\ \hline
    Maximum Cardinality & no     & yes          & yes        & yes               \\
    Pareto-Optimal      & yes    & yes          & yes        & yes               \\
    Popular             & no     & no           & no         & yes               \\
    Rank Maximal        & no     & no           & yes        & no                \\
    Always Exists       & yes    & yes          & yes        & no                \\
    Strategy Proof      & yes    & no           & no         & yes               \\ \hline
    Time Complexity     & $\mathcal{O}(n)$   & $\mathcal{O}(\sqrt{n}m)$ & $\approx\mathcal{O}(n^3)$    & $\mathcal{O}(\sqrt{C}n_1 + m)$ \\ \hline
    \end{tabular}
    \caption{Comparison of different algorithmic approaches}
    \label{tab:algorithm-comparison}
\end{table}

To summarize the results, we can see that all of the algorithms guarantee pareto-optimality, however only the Popular-CHA algorithm guarantees popularity. At the same time, only the greedy approach and Popular-CHA also guarantee strategy-proofness, which makes Popular-CHA particularly interesting for the student-seminar problem. 

\subsubsection{Strategy-proofness and maximum cardinality}
One interesting observation is that fulfilling maximum cardinality comes at the cost of either not being strategy-proof, or not guaranteeing that a matching exists at all. Indeed, only the greedy and Popular-CHA algorithm guarantee strategy-proofness. However, ensuring strategy-proofness and maximum cardinality at the same time comes at the cost of not always finding a matching. If we look back at the algorithm Popular CHA, we remember that a maximum cardinality matching $M'$ is computed on the reduced graph $G'$. We saw that a maximum popular matching does not exist, iff the matching is not agent-complete, meaning that one of the agents is matched to their last-resort house. While this mechanism ensures strategy-proofness, it is also not always possible to find such a maximum cardinality matching using the Popular CHA algorithm. Therefore, it remains an open question whether or not a mechanism exists that both is strategy-proof and produces maximum-cardinality matchings.

\subsubsection{Max-PaCHA and the assignment problem}
Another important thing to notice is the similarity of the properties between the Max-PaCHA and assignment problem algorithm. Except for the fact that the assignment algorithm guarantees rank maximality, the two algorithms produce matchings with very similar characteristics, which then begs the question why one should use the Max-PaCHA algorithm. But looking at the runtime complexity of the algorithms, we see that, while both algorithms run in polynomial time, the assignment problem takes longer to be solved.

\subsection{Practical results}