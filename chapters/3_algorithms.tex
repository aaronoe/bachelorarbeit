\section{Algorithmic Approaches}\label{chapter:algorithms}
In the previous section, we discussed several optimality criteria that apply to the problem of matching students to seminars. This chapter will present algorithms for computing matchings that fulfill some of those criteria, as well as evaluating them against each other. We will see that each of the algorithms has some draw-backs, which makes it necessary to consider the trade-offs when picking one for implementation. However, each of the algorithms also produces pareto-optimal matchings. 

\subsection{Greedy with Serial Dictatorship}\label{algo-rsd}
One of the simplest algorithms for the student-seminar matching problem is a greedy approach that iterates over the set of students and assigns each of the students to their most preferred seminar that still has some capacity left. In contrast to Gale \& Shapley's deferred acceptance algorithm for the stable marriage problem, this algorithm does not tentatively match students once they make their selection, but makes a final assignment for those students. Because of this, the algorithm finds a matching in $\mathcal{O}(n)$ time with $n$ being the number of students. This mechanism of letting students successively pick their highest available preference in order is known as serial dictatorship \cite{MANEA2007316}. In detail the algorithm is as follows: 

\begin{algorithm} % enter the algorithm environment
    \caption{Greedy serial dictatorship matching} 
    \label{alg1} % and a label for \ref{} commands later in the document
    \begin{algorithmic} % enter the algorithmic environment
        \Require set of Students with preferences $S$, set of Seminars $T$
        \Ensure Pareto-Optimal Matching M
        \Function{RSD-Matching}{$S, T$}
        \State $M := \emptyset$
        \ForEach {$s \in \mathcal S $ in random order}
            \State $t :=$ highest ranked, available seminar on preference list of s
            \If{$t \neq null$}
                \State $M = M \cup \{(s, t)\}$
            \EndIf
        \EndFor
        \State\Return M
        \EndFunction
    \end{algorithmic}
\end{algorithm}

Even though this algorithm is very simple and fast, it has some desirable properties, including one of the optimality criteria defined before.

\subsubsection{Properties}
Since the order, in which the students get to pick their match is pre-defined, we can easily show that the algorithm always produces a pareto-optimal matching.
\newtheorem{theorem}{Theorem}
\begin{theorem}
    A greedy algorithm that uses serial dictatorship always produces a pareto-optimal matching.
\end{theorem}
\begin{proof}
    Let $M$ be the matching produced by the algorithm. We assume that there exists a matching $N$ that pareto-dominates $M$. Now, let $s\in S$ be the first student who prefers his match in $N$ over $M$. Since $s$ prefers $N(s)$ over $M(s)$, the seminar $N(s)$ must have been unavailable when he made his pick. That means that another student $s' \in S$ exists, who picked $N(s)$ before $s$ could. However, we required that $s$ was matched to a better seminar in $N$, which means that $s'$ gets a worse match in $N$. This is a contradiction, so $N$ cannot pareto-dominate $M$ \cite{kun_2015}.
\end{proof}

We can also easily see that the algorithm is strategy-proof, because every applicant makes his final pick once it is his turn, there is no benefit in misrepresenting preferences \cite{Klaus}.

\subsubsection{Drawbacks}
When looking at the algorithm, it is clear that it has a strict preference order over students, specified by the order in which students are matched in the for-loop. Additionally, the algorithm makes no effort to match all students: if it is a student's turn to pick his match, and none of the seminars on his preference lists are free, that student will not be matched at all. This problem gets worse when we consider that our problem statement allows for incomplete preference lists, which increases the chances of having a high number of unmatched students. To illustrate this let us consider the following example in Table \ref{table:1}:

\begin{table}[h!]
    \centering 
    \begin{tabular}{llll}
    Agent   & Pref list     & Seminar   & Capacity \\
    $s_1$   & $t_1$, $t_2$  & $t_1$     & 1        \\
    $s_2$   & $t_1$         & $t_2$     & 1       
    \end{tabular}
    \caption{Instance where Serial Dictatorship admits no max cardinality matching}
    \label{table:1}
\end{table} 

In this example, each seminar only has a capacity of 1 and both students have seminar $t_1$ as their first preference. If the algorithm first gives $s_1$ a chance to pick, and then $s_2$, $s_1$ will be matched to $t_1$, making $t_1$ full and not allowing $s_2$ to be matched. On the other hand, matching $s_2$ to $t_1$ first and then matching $s_1$ to $t_2$ also yields a pareto optimal matching, however, in this case, all the students are matched.

To address the other problem of preference over students, a simple approach is using the random serial dictatorship mechanism, which instead creates a random order of students as the pick order.

\subsection{Pareto Optimal Maximal Matchings for CHA}\label{algo-max-po}
We have seen that serial dictatorship is an easy and time-efficient mechanism for computing Pareto-optimal matchings. A big weakness of the approach, however, is that it finds just one of many possible Pareto-optimal matchings without making any guarantees about quality in regards to cardinality. Particularly, the example in \mbox{Table \ref{table:1}} shows how permutations of the same instance can produce matchings of different cardinality, which motivates the search for an algorithm that produces a Pareto-Optimal matching of maximum cardinality. 

Abraham et. al \cite{Abraham:Pacha} have proposed a 3-phase algorithm for computing a maximum cardinality matching for the house allocation problem, which was extended by Sng \cite{SngThesis} for the many-to-one case (CHA). Before presenting the algorithm, an important lemma about Pareto optimal matchings has to be shown first, which is then used to prove the correctness of the algorithm. To characterize the lemma, we need to define the terms \emph{maximality, trade-in-free and cyclic coalition} in regards to a matching $M$ first: 
\begin{enumerate}
    \item \textbf{Maximal}: $M$ is maximal, if no student $s_i \in S$ and seminar $t_j \in T$ exists, so that $s_i$ is unassigned, $t_j$ is undersubscribed in $M$ and $t_j$ is on $s_i$'s preference list \cite{Abraham:Pacha}.
    \item \textbf{Trade-in-free}: $M$ is trade-in-free, if there are no student $s_i \in S$ and seminars $t_j, t_l \in T$, such that $s_i$ is assigned to $t_l$, but prefers $t_j$ over $t_l$ and $t_j$ is undersubscribed \cite{Abraham:Pacha}. 
    \item \textbf{Cyclic coalition}: $M$ contains a cyclic coalition, if there exists a sequence of distinct assigned students $C = \langle s_0, s_1, \dots, s_{r-1} \rangle$ with $r \geq 2$, such that $s_i$ prefers $M(s_{i + 1 \bmod r})$ (i.e. the seminar assigned to the next student in $C$ after $s_i$) over $M(s_i)$ for every $i$ \cite{Abraham:Pacha}.
\end{enumerate}
Using these definitions, Sng now presents and proofs the following lemma:
\newtheorem{lemma}[theorem]{Lemma}
\begin{lemma}\label{lemma-pacha}
    Let $M$ be a matching of a given instance of CHA. Then $M$ is Pareto optimal if and only if $M$ is maximal, trade-in-free and cyclic-coalition-free \cite{Abraham:Pacha}.
\end{lemma}

Using this lemma, the authors \cite{SngThesis, Abraham:Pacha} construct a 3-phased algorithm, where each phase fulfills one of the properties as described in Lemma \ref{lemma-pacha}, like so: Let $I$ be an instance of CHA and $G$ it is underlying graph, then perform the following steps:
\begin{enumerate}
    \item \textbf{Phase 1}: In order to guarantee maximality, compute a maximum-cardinality matching $M$ in $G$.
    \item \textbf{Phase 2}: Using the matching $M$ produced by step 1, the algorithm then fulfills the trade-in-free criteria as follows: Search for pairs $(s_i, t_j) \in M$ with $s_i \in S$ and $t_j \in T$ and where $t_j$ is undersubscribed in $M$ and $s_i$ prefers $t_j$ over his own match $t_l := M(s_i)$. Whenever such a pair is found, remove the existing assignment $(s_i, t_l)$ and add $(s_i, t_j)$ to $M$. Consequently $t_l$ is now undersubscribed and may be assigned to another student. Therefore, we continue the search for such pairs until no such pair can be found for every student in $S$.
    \item \textbf{Phase 3}: The last phase of the algorithm eliminates any cyclic coalitions from $M$, if they exist, by using a modified version of \emph{Gale's Top Trading Cycles} (denoted by TTC) Method \cite{ShapleyTTC}. Essentially, the TTC method creates a graph from the matching $M$, where every student that is not matched to his most-preferred seminar, denoted by $S'$, is represented by a node. Next, a directed edge is created from each student $s_i \in S'$, to all students in $S'$ who are assigned to the first seminar on $s_i$'s preference list. Now, there must be at least one cycle in this graph, as students may have an edge to themselves. The next step is identifying the cycles and implementing a trade among all agents of that cycle that reassigns the seminars among these students. After the trade, all students from that cycle are removed and these steps are repeated until the graph is empty. Once the graph is empty, $M$ is coalition-free by the correctness of the TTC method \cite{Abraham:Pacha}.
\end{enumerate}

\subsubsection{Properties}
Since all modifications to the matching in phase 1 and 2 are limited to swaps and no deletions, maximum cardinality is still guaranteed after the termination of phase 3. Additionally, the resulting matching is also trade-in-free and cyclic-coalition-free as those properties are guaranteed after performing phase 2 and 3 respectively. However, it is important to note that this algorithm, unlike the serial dictatorship mechanism, is not strategy-proof. Due to the fact that a maximum-cardinality matching is computed in step 1, students are encouraged to provide short preference lists to have a higher chance of being matched to their first preference. Forcing students to provide complete preference lists could make this mechanism strategy-proof, if we can assume that students do not have any knowledge of the other student's preferences.

The runtime of the algorithm is dominated by finding a maximum cardinality matching (phase 1), which yields a time complexity of $\mathcal{O}(E\sqrt{V})$ \cite{Abraham:Pacha} when using the Hopcroft-Karp algorithm. Phase 1 and 2 both take $\mathcal{O}(|E|)$ of time \cite{SngThesis}, which in total yields a worst-case complexity of $\mathcal{O}(E\sqrt{V})$ for finding a maximum cardinality pareto-optimal matching given any instance $I$ of the problem.

\subsection{Solving CHA as an Assignment Problem}\label{algo:assignment}
In order to find a rank-maximal matching, we will investigate a set of algorithmic methods that compute a maximum-cardinality, min-weight matching. We can easily see that such a matching must also be rank-maximal, since the min-weight property guarantees that the profile of the matching is lexicographically maximal among all matchings.
In Section \ref{intro_assignment}, we briefly presented the assignment problem: it is a combinatorial optimization problem, which assigns a set of agents to a set of tasks, wherein each agent-task tuple is assigned a cost, and to minimize the total cost of the assignment. 

% TODO: maybe move this somewhere

Formally, we define the assignment problem as follows: Given a set of agents $A$, tasks $T$ and a map $W(a, t) = w$, with $\forall a \in A, \forall t \in T$, find a bijective map $M$ with: $\forall a \in A, \exists t \in T: M(a) = t$, so that the following objective function is minimized: $\sum_{a \in A} W(a, M(a))$. 

It is important to note that the assignment problem tries to find a perfect matching, meaning that all agents are assigned, and that a one-to-one (HA), contrary to one-to-many (CHA), matching is found.

For this problem, there exist algorithms \cite{Munkres, Jonker1987} that find perfect, min-weight assignments in $\mathcal{O}(n^3)$ time. Using these algorithms for the one-to-many case with incomplete preference lists requires a transformation of the input, by introducing artificial edges with large weights, where no edges exist. Additionally, in the case of student-seminar matchings, seminars have to be duplicated according to their capacities to allow for performing one-to-many matchings.
  
However, we can see that the assignment problem is equivalent to finding a perfect, minimum-weight matching in bipartite, weighted graph. Therefore, we can construct such a graph, given the set of students, preference lists and seminars and apply graph algorithms that find such matchings for us. In fact, we can simply transform the student-seminar matching problem into an instance of the minimum-cost flow problem. The goal of this algorithm will be to send $|S|$ units of flow through the network, while minimizing the cost of the flow, which is indicated by a seminar's rank on the student's preference lists.

\subsubsection{Solving the Assignment Problem Using Flow-Networks}
The problem of matching students as seminars can be given as a bipartite graph $G=(V=(S, T), E)$, where $S$ is the set of students and $T$ the set of seminars. The set of edges $E$ is defined as follows: $E:= \{(s, t) \mid s \in S \land t \in T \land \mbox{t is on the preference list of s}\}$. Additionally, a weight function $W: E \rightarrow  \mathbb{N}$ is specified, which maps each edge to the position of the seminar on the student's preference list. In order to transform this bipartite graph into an input for the minimum-cost flow problem, a flow network has to be constructed using the bipartite graph.

To transform the bipartite graph into a flow network, we first add a source and sink vertex to the graph. Then, we add weights, capacities and edges from and to the source and sink. Specifically, we create an edge from the source to each of the student vertices with a capacity of 1 and a cost of 0. These edges indicate that a student can only be assigned once. Next, for every student we re-use the edges from the bipartite graph $G$, where each edge $e \in E$ is assigned a capacity of 1 and a cost of $W(e)$. The capacity, again indicates that a student can only be assigned once and the weight indicates the position of the seminar on the student's preference list. Finally, one edge is added from each seminar $s \in S$ to the sink with a capacity of $C(s)$ and a cost of 0. 

\subsubsection{Properties}
The matching $M$ computed by the algorithm is Pareto optimal \cite{SngThesis} and has the minimum weight property, and therefore is rank-maximal;\cite{SngThesis} however, a large drawback is that this mechanism is not strategy-proof. Students are again encouraged to provide short preference lists in order to get matched to their most-preferred seminar. The algorithm tries to match every student, which means that students with a list of just one seminar will be prioritized over students, who prefer the same seminar but also supply other preferences. This problem could lead to having all students provide single-element preference lists, which increases the difficulty of finding perfect matchings. Similarly to the previous algorithm, requiring complete preference lists could alleviate this problem and make the mechanism strategy-proof, if it can be assumed that students' preferences are not publicly known.

\subsection{Maximum Popular Matchings in CHA}\label{algo-max-pop}
Abraham et al \cite{AbrahamPopular} presented an algorithm for finding a popular matching in the house allocation problem (without capacities), which either finds such a popular matching or reports that none exists in $\mathcal{O}(|V| + |E|)$ time. Manlove and Sng \cite{ManlovePopularMatchings} extended this algorithm for the many-to-one case, the Capacitated House Allocation problem, by developing a characterization of such popular matchings in CHA and then using it to construct an algorithm that finds a maximum popular matching for any given instance, if it exists. 
Intuitively, the algorithm will try to match as many applicants as possible to their most-preferred choice to fulfill the popularity criteria. Formally, Manlove and Sng \cite{ManlovePopularMatchings} define and proof an alternative characterisation of Popularity in order to develop their algorithm:

\subsubsection{An Alternative Characterization of Popular Matchings}
Given an instance $I$ of the CHA problem, for every applicant $a_i \in A$ let $h_j := f(a_i)$ be the first ranked house on $a_i$'s preference list. In this case, we call $h_j$ an f-house. For each house $h_j \in H$, define the set of applicants, who named $h_j$ as their first choice, as $f(h_j) = \{a_i \in A: f(a_i) = h_j\}$ and the size of that set as $f_j = |f(h_j)|$. For a matching $M$ in $I$, we now say that a house $h_j \in H$ is full if $|M^{-1}(h_j)| = c_j$, i.e. the maximum number of applicants is matched to that house, and undersubscribed if $|M(h_j)| < c_j$. Additionally, for every applicant $a_i \in A$, we append a last-resort house $l(a_i)$ with capacity 1 to $a_i$'s preference list \cite{ManlovePopularMatchings}. The authors prove the following lemma \cite{ManlovePopularMatchings}:
\newtheorem{lemma-popular-1}[theorem]{Lemma}
\begin{lemma}\label{lemma-popular1}
    Let $M$ be a popular matching in $I$. Then for every f-house $h_j$, $|M^{-1}(h_j) \cap f(h_j)| = \min\{c_j, f_j\}$.
\end{lemma} 
In other words, for a popular matching $M$ in $I$, every f-house $h_j$ is matched to at least the number of applicants who have $h_j$ as their first-preference but at most $h_j$'s capacity $c_j$. Next, for every applicant $a_i$, we define $s(a_i)$ to be the most-preferred house $h_j$ on his preference list, such that either (i) $h_j$  is not an f-house, meaning that it is not the first choice of any applicant, or (ii) $h_j$ is an f-house, but $h_j \neq f(a_i)$ and $f_j < c_j$. In simple terms, $s(a_i)$ is the first undersubscribed house on $a_i$'s preference list after $f(a_i)$. We will refer to such houses as s-houses. It is important to note that such a house $s(a_i)$ always exists, due to the introduction of $l(a_i)$. In a popular matching, an agent $a_i$ may only be matched to either $f(a_i)$ or $s(a_i)$, as every house in between those two is full according to the defintion of $f(a_i)$ and $s(a_i)$. Manlove and Sng, again prove the following two lemmas \cite{ManlovePopularMatchings}: 

\newtheorem{lemma-popular-2}[theorem]{Lemma}
\begin{lemma}\label{lemma-popular2}
    Let $M$ be a popular matching in $I$. Then no agent $a_i \in A$ can be matched in $M$ to a house between $f(a_i)$ and $s(a_i)$ on $a_i$'s preference list.
\end{lemma} 

\newtheorem{lemma-popular-3}[theorem]{Lemma}
\begin{lemma}\label{lemma-popular3}
    Let $M$ be a popular matching in $I$. Then no agent $a_i \in A$ can be matched in $M$ to a house worse than $s(a_i)$ on $a_i$'s preference list.
\end{lemma} 

To summarize, so far for every applicant $a_i \in A$ we have defined the applicant's most preferred house $f(a_i)$ and his second most-preferred, but available house $s(a_j)$. We have seen that, in a popular matching, applicants can only be matched to either of those houses $f(a_i)$ or $s_(a_i)$. Using this information and a description of an instance as a weighted-bipartite-graph $G = (V=(A, H), E)$, we can now construct a subgraph $G'$ of $G$, by removing all edges in $G$ from every applicant $a_i$, except the ones to $f(a_i)$ and $s(a_i)$. We now say that a matching $M$ is agent-complete in $G'$ if it matches all agents in $A$ and no agent $a_i$ is matched to their last-resort house $l(a_i)$ \cite{ManlovePopularMatchings}. Manlove and Sng prove the following theorem to fully characterize popular matchings \cite{ManlovePopularMatchings}:
\newtheorem{theorem-popular-4}[theorem]{Theorem}
\begin{theorem}\label{theorem-popular-4}
    A matching $M$ is popular in $I$ iff:
    \begin{enumerate}
        \item for every f-house $h_j$
        \begin{enumerate}
            \item\label{condition1a} if $f_j \leq c_j$, then $f(h_j) \subseteq M(h_j)$
            \item\label{condition1b} if $f_j > c_j$, then $|M(h_j)| = c_j$ and $M(h_j) \subseteq f(h_j)$
        \end{enumerate}
        \item $M$ is an agent complete matching in the reduced graph $G'$
    \end{enumerate}
\end{theorem} 

Intuitively, this means that in every popular matching, each house that would be undersubscribed or just full, based on the applicants who name that house as their first preference, will be matched to at least all of those applicants. Additionally, every house that would be oversubscribed is matched to only a subset of the applicants naming the house as their first choice. For the remaining unmatched applicants it must now be possible to find an applicant-complete matching so that none of the applicants is matched to their last-resort house.

\subsubsection{Algorithm}
Using Theorem \ref{theorem-popular-4}, the authors develop the algorithm Popular-CHA for finding a maximum popular matching or reporting that none exists \cite{ManlovePopularMatchings}. The algorithm works as follows:
\begin{enumerate}
    \item Reduce $G$ to $G'$.
    \item Match all agents to their first-choice house $h_j$, if $f_j \leq c_j$, i.e. the house would be undersubscribed or just full afterwards. This will satisfy condition \ref{condition1a} of Theorem \ref{theorem-popular-4}.
    \item Remove all applicants and their incident edges, that were matched in the previous step, from $G'$. Additionally, update the capacities for each previously matched house $h_j$ as $c'_j = c_j - f_j$. All full and isolated houses and their incident edges are also removed from $G'$.
    \item Compute a maximum cardinality matching $M'$ on $G'$ using the updated capacities. For this step Manlove and Sng use Gabow's algorithm \cite{Gabow1983}.
    \item If $M'$ is not agent complete, then no popular matchings exists. Otherwise merge the matchings $M$ and $M'$.
    \item As a last step, to fulfill condition \ref{condition1b} of Theorem \ref{theorem-popular-4}, promote any agent $a_i \in M$ who is matched to their s-house to their f-house, if it is undersubscribed. 
\end{enumerate}


\subsubsection{Properties}
Due to the fact that step 2, 4 and 6 make the computed matching fulfill the criteria outlined in Theorem \ref{theorem-popular-4}, the algorithm produces a popular matching of maximum cardinality, if it exists. If just one applicant can't be matched to his f-house or s-house or is matched to his last-resort house, the algorithm fails and no matching is produced. In that case the popular matching for that instance is not of maximum cardinality and therefore is not computed by the algorithm. 

Furthermore, the algorithm's runtime complexity is $\mathcal{O}(\sqrt{C} * n_1 + |E|)$, where $C$ is the sum of the capacities of the houses and $n_1$ the number of applicants. $|E|$ is equivalent to the sum of the agents' preference list lengths. The runtime is dominated by Gabow's algorithm \cite{Gabow1983}, which computes the maximum cardinality matching in $G'$ in $\mathcal{O}(\sqrt{C}n_1)$ \cite{ManlovePopularMatchings}. Alternatively, a modified version of the Hopcroft-Karp algorithm could be used for computing the maximum cardinality matching in $\mathcal{O}(E\sqrt{V})$\cite{Hopcroft} time, which is what Abraham et al. use for the one-to-one case.

If the applicants do not have knowledge over the overall preference distribution, this mechanism is strategy-proof due to the existence of the l-houses. The mechanism does not prioritize students with short preference lists, because it only considers at most two of the students preferences. However, if students provide short preference lists, it is more likely for this algorithm to not find a matching, as the chances of a student being matched to his last-resort house are increased.